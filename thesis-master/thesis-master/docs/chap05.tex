%%
% 结论
% 结论是毕业论文的总结,是整篇论文的归宿,应精炼、准确、完整。结论应着重阐述自己的创造性成果及其在本研究领域中的意义、作用,还可进一步提出需要讨论的问题和建议。
% modifyer: 黄俊杰(huangjj27, 349373001dc@gmail.com)
% update date: 2017-04-13
%%

\chapter{总结与展望}
本文提出了一种基于机器人学习经验的基于机器人的自适应控制框架。 我们将Z轴速度作为反馈信号,并采用回归来纠正机器人的动作。 我们通过聚类简化运行时学习,并将多元回归转换为单元回归。 实验表明该方案是有效的。 值得注意的是,该方法不仅可以用于本实验中的攀爬管,还可以用于其他机器人应用。 我们相信该算法可以适应其他相应的场景,如无人驾驶车辆的可变运动,蛇形机器人的粗糙地面运动和模拟PID控制,只要给出足够的训练数据和清晰的运动目的。