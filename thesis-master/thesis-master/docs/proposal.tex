%%
% 开题报告
% modifier: 黄俊杰(huangjj27, 349373001dc@gmail.com)
% update date: 2017-05-14

% 选题目的
\objective{
	\qquad 蛇形机器人是一类仿生超冗余度机器人,用于模拟蛇类生物在野生环境中的仿生运动,具有快速、稳定、多样性等特点。这些蛇形机器人多数是由许多活动关节模块连接组成,从而赋予它们运动学上的多功能性,如弯曲、伸展和盘绕。蛇形机器人已经应用在各个领域,如救灾、工厂维护和恐怖主义的监视任务。
	
	\qquad 为了更好地完成预定义的任务,蛇形机器人需要获得自主移动和自适应行为的能力,例如,根据其所处环境的不同情况决定何时、何地以及如何移动来自动适应环境,然而这种实现是并不容易的。首先,由于蛇形机器人具有冗余的自由度,使得机器人的运动建模比较困难,尤其是机器人与环境之间复杂的相互作用,使得多自由度机器人的运动控制更加复杂。第二,当遇到事先未知的环境时,显而易见的需要确定可控制的策略。即使对于给定的控制策略,如何确定其参数也不简单。第三,确定控制策略及其相应参数的运行时过程必须是有效的,即实时的。否则,机器人的期望运动将很可能失败。
	
	\qquad 为了让机器人适应未知环境,利用传感器感知环境和嵌入环境感知规则的方法已经被广泛使用。提出了一种基于CPG(中心模式发生器)模型的控制策略。提出了一种基于状态估计的蛇形机器人自适应控制。由于这些方法中的模型是基于状态估计的梯度模型。有研究者提出了结合物理环境信息的神经网络模型来确定控制方案,然而这些模型仅仅是控制建议。
	
	\qquad 所以本课题提出一种新的蛇形机器人自适应控制框架。基于离线无监督训练收集的数据,我们的框架能够通过快速回归有效地适应环境变化的新的控制参数的值。具体来说,我们的目标是爬杆,这需要快速响应环境的变化即杆的半径的变化。在这种情况下,如果机器人的运动不能做到及时适应,机器人就会掉下来。为了评价该方法的有效性,我们计划对蛇形机器人在不同杆上进行了实验以及对于不同爬杆步态采用算法进行实验。
	

}

% 思路
\methodology{
	\qquad 通过数据驱动的方式,基于蛇形机器人爬杆时所用的步态——Rolling步态,主要分成两大步骤:
	\begin{enumerate}
		\item OFFLINE训练:确认参数的变化空间,然后进行离线的训练,采集训练数据(如关节角度,参数值,速度等)。然后对采集的大量数据进行一个预处理,使用聚类方法进行归类。
		\item ONLINE算法驱动:通过实时地按照一定频率不断调用算法来调整蛇形机器人的步态参数值。通过利用回归分析来获取所需要修改的参数值来调整蛇形机器人的步态。
	\end{enumerate}
}

% 研究方法/程序/步骤
\researchProcedure{
	\qquad 本研究采用的研究方法如下:
	\begin{enumerate}
		\item 相关论文查找,归类,阅读及综合分析;
		\item 实验环境的搭建(仿真模型的建模或者实物机器人的构建和检测);
		\item 算法代码的编写实现;
		\item 进行样例实验检测算法的可行性和对比实验检测算法的有效性;
		\item 收集实验数据进行实验结果的对比分析;
	\end{enumerate}
}

% 相关支持条件
\supportment{
	\qquad 实验所需的软硬件为:
	\begin{enumerate}
		\item V-REP仿真平台;
		\item 带有Linux操作系统的主机一台;
		\item 蛇形机器人实物;
		\item 各种元件模块如wifi,芯片等;
	\end{enumerate}
}

% 进度安排
\schedule{
	\qquad 整体进度安排如下:
	\begin{itemize}
		\item 2018.12-2019.01 进行文献查阅,确认算法实现框架;
		\item 2019.01-2019.02 进行实验环境的搭建和算法代码的编写实现;
		\item 2019.02-2019.03 进行样例实验以及对比实验,采集实验数据;
		\item 2019.03-2019.04 进行实现结果分析以及学位论文初稿的撰写;
		\item 2019.04-2019.05 进行算法或者实验的优化,在指导老师的意见下对学位论文进行更进一步地修改;
		\item 2019.05-2019.06 准备毕业论文答辩;
	\end{itemize}
}

% 指导老师意见
\proposalInstructions{

}

