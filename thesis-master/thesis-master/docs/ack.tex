%%
% 致谢
% 谢辞应以简短的文字对课题研究与论文撰写过程中曾直接给予帮助的人员(例如指导教师、答疑教师及其他人员)表示对自己的谢意,这不仅是一种礼貌,也是对他人劳动的尊重,是治学者应当遵循的学术规范。内容限一页。
% modifier: 黄俊杰
% update date: 2017-04-15
%%

\chapter{致谢}

%	四年时间转眼即逝,青涩而美好的本科生活快告一段落了。回首这段时间,我不仅学习到了很多知识和技能,而且提高了分析和解决问题的能力与养成了一定的科学素养。虽然走过了一些弯路,但更加坚定我后来选择学术研究的道路,实在是获益良多。这一切与老师的教诲和同学们的帮助是分不开的,在此对他们表达诚挚的谢意。
%
%	首先要感谢的是我的指导老师林倞教授。我作为一名本科生,缺少学术研究经验,不能很好地弄清所研究问题的重点、难点和热点,也很难分析自己的工作所能够达到的层次。林老师对整个研究领域有很好的理解,以其渊博的知识和敏锐的洞察力给了我非常有帮助的方向性指导。他严谨的治学态度与辛勤的工作方式也是我学习的榜样,在此向林老师致以崇高的敬意和衷心的感谢。
%
%	最后我要感谢我的家人,正是他们的无私的奉献和支持,我才有了不断拼搏的信息的勇气,才能取得现在的成果。

	四年的大学时光真的在不经意间悄然就到尽头了,一段美妙难忘的本科大学生活即将画上句号。从此我将告别象牙塔的生活,去社会接受检验和磨练。回顾大学四年的时光,虽然走的并不是一帆风顺和轻松惬意,但是却在克服一个个困难之后收获了许多,成长了许多。在此毕业论文即将付梓之际,我想在这感谢老师四年的教诲,家人的关心鼓励以及同学的帮助。
	
	首先最想感谢的是我的导师,黄凯老师。黄凯老师在我看来在科研上是一个科学严谨,一丝不苟的学者。在私底下是一个幽默风趣,平易近人的前辈。在学习上,他耐心地,严谨地指导我的科研项目,指出我的不足,解答我的疑问。在生活中,老师也无时无刻不在关心着我们的学习和生活。从黄凯老师身上我学会了许多,也收获了许多。
	
	其次我想对我的父母亲人说一声谢谢,他们是我最坚实的后盾,没有他们的督促我可能在学业这条路上走不了这么远,没有他们的支持我可能无法在很多事情上勇敢地做出决定。然后我还想谢谢我的女朋友,她的出现带给了大三迷茫的我为自己梦想而拼搏奋斗的勇气,她的陪伴和鼓励给予我不断向前的勇气和信息。在这里我也想对正在学业道路上奋斗的她说一声:“相信自己,你一定可以到达成功的彼岸的!”
	
	最后我还要感谢下陪伴我渡过大学时光同学挚友们,我们在学业生活中互相砥砺前行,互相扶持,感谢你们,让我在前进的路上不会孤单,感谢你们在我向你们伸出援手的时候无私地帮助我。还要感谢一直和我联系的小学,初中,高中同学,感谢你们一直的陪伴,给予我深刻和真诚的友谊,望日后我们都能够活出自己,拥抱未来。

\vskip 108pt
\begin{flushright}
	简智勇\makebox[1cm]{} \\
\today
\end{flushright}

