%%
% 引言或背景
% 引言是论文正文的开端,应包括毕业论文选题的背景、目的和意义;对国内外研究现状和相关领域中已有的研究成果的简要评述;介绍本项研究工作研究设想、研究方法或实验设计、理论依据或实验基础;涉及范围和预期结果等。要求言简意赅,注意不要与摘要雷同或成为摘要的注解。
% modifier: 黄俊杰(huangjj27, 349373001dc@gmail.com)
% update date: 2017-04-15
%%

\chapter{引言}
%定义,过去的研究和现在的研究,意义,与图像分割的不同,going deeper
\label{cha:introduction}
\section{选题背景与意义}
\label{sec:background}
% What is the problem
% why is it interesting and important
% Why is it hards, why do naive approaches fails
% why hasn't it been solved before
% what are the key components of my approach and results, also include any specific limitations,do not repeat the abstract
%contribution
蛇形机器人是一类具有高冗余度的仿生机器人~\cite{Chirikjian1995The}。蛇形机器人设计核心在于模仿生物蛇的独特的无肢运动以及生物蛇在野外表现出的出色的快速,稳定,多样的运动能力和方式。仿生蛇形机器人通常是由许多可活动关节模块连接组成,这种组成方式可赋予蛇形机器人多样的运动形式,例如弯曲和伸缩,蛇形机器人的模块间结构给蛇形机器人像生物蛇一样做蜿蜒运动,侧移运动或者伸缩运动提供了结构基础。一定数量的蛇形机器人已经应用在各个领域,例如灾难的救援(地震之后的救援探测),工厂管道的维护和应用于恐怖事件中作监视作用。

为了更好的完成预定义的任务,蛇形机器人需要获得自主移动和自适应环境及调整自身行为运动的能力~\cite{Liljeb2013Snake},例如,根据自身和环境的不同情况决定何时何地以及如何移动。然而,自适应环境去调整自身的运动并不是一件简单的事情。原因是多方面的。首先,冗余的自由度给蛇形机器人的运动建模制造了很大的困难,特别是在机器人和环境之中的复杂交互中建模尤为困难。多自由度的运动对应的控制方案甚至会更加的复杂。其次,当遇到未知的环境时,如何确定合适的控制策略显然是一个关键性的问题。即使对于一个给定的策略,如何确定控制参数也不是可以一步到位去确定一个具体的参数或者数值的。第三,决定控制策略及其相应的参数必须在实时的运动过程中做到有效及迅速即时,否则蛇形机器人的运动很有可能会失败甚至导致蛇形机器人的损坏。

本文提出了一种新的蛇形机器人的自适应控制模型。本文模型基于无监督训练离线收集经验数据,然后通过快速回归出新的控制参数值来有效地适应环境的变化。具体来说,本文实验以在爬杆过程中的对环境变化的机器人的快速响应为目标。因为在爬杆过程中,如果控制策略不能做到及时,准确,那么机器人就会出现从杆上掉落的情况。为了评估本文模型的有效性,我们让蛇形机器人在我们的控制策略控制下在不同直径的杆上进行攀爬运动。实验结果表明,运行时的算法开销平均值略大于100\,ms并且蛇形机器人可以快速自适应在不同直径的杆上的攀爬运动,顺利沿着杆爬升。本文的意义在于:
\begin{itemize}
	\item 提出了一个蛇形机器人爬杆的自适应控制模型。 对于离线无监督训练,使用k-means++方法对收集的数据进行聚类。 在运行期间,将机器人的Z轴速度用作反馈信号,以不断调整机器人的新控制参数。
	\item 将一个多元回归模型简化成一个一元回归模型来实时计算最优的控制参数值,通过熵方差实时去选出当前状态下的最敏感控制参数。然后通过仅对这个参数做回归计算,将一个多参数回归模型转换成了一个一元回归模型。
	\item 经过多组实验表明,本文提出的控制策略实现了蛇形机器人自主运动的有效性。
\end{itemize}

\section{国内外研究现状和相关工作}
\label{sec:related_work}

为了让蛇形机器人适应未知的环境,能够在未知环境中做出正确的运动姿态,使用嵌入式传感器来感知环境并根据传感器数据做出对应的环境感知之后做出相对应的控制算法的方法已被广泛使用~\cite{BalancingAndControl}~\cite{FeedbackControlOfSoft}~\cite{CPGenabling}~\cite{GaitBasedCompliant}。唐超权等提出了一种基于多模态CPG(中枢模式发生器)模型的控制策略~\cite{CPGenabling},该控制策略是基于预定义的预期速度值,利用当前速度和预期速度的关系来自调整输入信号进而诱导CPG模型做出运动的改变的多模态CPG模型。Rollinson等提出了一种基于状态估计的蛇形机器人自适应控制方案~\cite{GaitBasedCompliant}。通过EKF(扩展卡尔曼滤波)整合传感器信息在每个时间步长处有效地估计基于步态的蛇形机器人状态,然后选择相对于该状态的参数控制空间来控制机器人的运动。 以上两种控制方案的提出在相应场景下对机器人的控制比人类去单独控制每一个控制参数来控制机器人以及固有的运动方程控制方案确实更有效,且能运用更丰富灵活的步态运动。然而参数化步态和中央模式生成器~\cite{ijspeert2008central}这样的简单控制器虽然已经取得了一些成功,但使用这些控制器创建自主或自适应行为已经证明是困难的。同时对于自适应运动我们也要考虑效率上的问题。在如今人工智能的浪潮下,对于机器人的自适应控制,有研究人员提出神经网络模型结合物理环境信息来确定控制方案~\cite{InformationDriven}~\cite{NovelPlasticityRule}~\cite{MissileSystems}~\cite{NeuroFuzzyBayesian},但是目前还无法通过具体实际的实验来验证。


\section{本文的论文结构与章节安排}

\label{sec:arrangement}
本文共分为五章,各章节内容安排如下:

第一章:引言,本章介绍了蛇形机器人在自适应运动控制方面的研究背景,意义,研究现状及相关工作。

第二章:蛇形机器人结构和控制方法,介绍了本文中采用的蛇形机器人设计结构以及基于正交结构设计的蛇形机器人常用控制函数。

第三章:研究方法介绍,详细地介绍了本文所采用的研究算法,介绍了算法的框架以及每个步骤的详细分析。

第四章:实验和结果展示,展示了本文所采用的算法在蛇形机器人爬杆运动中的表现情况,包括数据分析,对照实验,算法性能分析。

第五章:总结与展望。总结本文的研究结论,同时根据目前理论和实验的不足之处,对下一步的研究工作做展望。

