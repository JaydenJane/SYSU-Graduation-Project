%%
% 引言或背景
% 引言是论文正文的开端,应包括毕业论文选题的背景、目的和意义;对国内外研究现状和相关领域中已有的研究成果的简要评述;介绍本项研究工作研究设想、研究方法或实验设计、理论依据或实验基础;涉及范围和预期结果等。要求言简意赅,注意不要与摘要雷同或成为摘要的注解。
% modifier: 黄俊杰(huangjj27, 349373001dc@gmail.com)
% update date: 2017-04-15
%%

\chapter{引言}
%定义,过去的研究和现在的研究,意义,与图像分割的不同,going deeper
\label{cha:introduction}
\section{选题背景与意义}
\label{sec:background}
% What is the problem
% why is it interesting and important
% Why is it hards, why do naive approaches fails
% why hasn't it been solved before
% what are the key components of my approach and results, also include any specific limitations,do not repeat the abstract
%contribution
蛇形机器人是一种具有超高冗余度的仿生机器人~\cite{Chirikjian1995The}。蛇形机器人设计核心在于模仿生物蛇的独特的无肢运动以及生物蛇在野外表现出的出色的快速,稳定,多样的运动能力和方式。仿生蛇形机器人通常是由许多可活动关节模块连接组成,这种组成方式赋予了蛇形机器人多样的运动形式,例如弯曲和伸缩,蛇形机器人的模块间结构给蛇形机器人像生物蛇一样做蜿蜒运动,侧移运动或者伸缩运动提供了结构基础。蛇形机器人已经应用在各个领域,例如灾难的救援(地震之后的救援探测),工厂管道的维护和应用于恐怖事件中作监视作用。

为了更好的完成预定义的任务,蛇形机器人需要获得自主移动和自适应环境及调整自身行为运动的能力~\cite{Liljeb2013Snake},例如,根据自身和环境的不同情况决定何时何地以及如何移动。然而,自适应环境去调整自身的运动并不是一件简单的事情。原因是多方面的。首先,高冗余的自由度给蛇形机器人的运动建模制造了很大的困难,特别是在机器人和环境的复杂交互中建模尤为困难。拥有多自由度的蛇形机器人结构使得为其设计对应的控制方案更加的复杂。其次,当遇到未知的环境时,如何确定合适的控制策略显然是一个关键性的问题。即使对于一个给定的策略,如何确定控制参数的值来使得蛇形机器人能够做出期望的运动也不是可以一步到位的。第三,决定控制策略及其相应的参数必须在实时的运动过程中做到有效、迅速、及时,否则蛇形机器人的运动很有可能会失败甚至导致蛇形机器人的损坏。

本文提出了一种新的蛇形机器人在自适应运动中的控制模型。具体来说,本文实验以在爬杆过程中蛇形机器人对环境变化的快速响应为目标。模型基于无监督训练离线收集经验数据,然后通过快速回归分析得到新的控制参数值来控制蛇形机器人有效地适应环境的变化。因为在爬杆过程中,如果控制策略不能做到及时,准确,那么机器人就会出现因为杆的直径发生变化从而从杆上掉落的情况。为了评估本文模型的有效性,本文让蛇形机器人在本文提出的控制策略控制下在不同直径的杆上进行攀爬运动。

本文的工作包括如下几个方面:
\begin{itemize}
	\item 提出了一个蛇形机器人自适应爬杆运动的控制模型。 通过将离线无监督训练和机器人运行时的算法结合,蛇形机器人能够不断调整控制参数来让运动顺利进行。
	\item 通过熵方差作为选择指标,将一个多元回归分析模型简化成一个一元回归分析模型来实时计算最优的控制参数值。
	\item 通过分析算法复杂度以及实验结果,运行时的算法开销平均值略大于100\,ms并且蛇形机器人可以快速自适应在不同直径的杆上的攀爬运动,顺利在杆上做攀爬运动。
\end{itemize}

\section{国内外研究现状}
\label{sec:related_work}
\subsection{国外研究现状}
Rollinson等提出了一种基于状态估计的蛇形机器人自适应控制方案~\cite{GaitBasedCompliant}。通过EKF(扩展卡尔曼滤波)整合传感器信息在每个时间步长处有效地估计基于步态的蛇形机器人状态,然后选择在该状态下相对应的控制参数空间来控制蛇形机器人的运动。

Kuwada等提出了管道内行进的蛇形机器人的自动行进方法~\cite{Kuwada2008Automatic}。利用每个关节中的智能执行器,包括直流电机和微处理器,开发了夹持力控制,管径自适应控制和管道曲线的自适应控制三种控制方法从而使得蛇形机器人很好地在不同直径的管道内,包括在弯道管道,T型管道和垂直管道中自动行进。

Vitiello等提出了一种基于神经模糊贝叶斯系统的机器人自适应空间行为的控制方法~\cite{NeuroFuzzyBayesian}。通过这个神经网络系统,机器人能够根据人类用户的个性和人类用户的目前的行为自适应地调整机器人本身的空间位置和行为。通过创新协同的方法结合模糊逻辑,神经网络和贝叶斯分类器来调整机器人关于人类个性和活动的行为(与人类交互时距离人类对象的停止距离)。

Nachstedt等针对于具有螺旋驱动结构的蛇形机器人提出了使用四个具有突触可塑性的自适应神经振荡器作为CPG(中央模式发生器)来通过螺旋驱动机制实现蛇形机器人的自适应运动~\cite{Nachstedt2013Adaptive}。通过使用基于频率适应和Hebbian型学习规则的自适应机制,仅有三个神经元组成的振荡器能够自动生成适合蛇形机器人运动的周期性模式并且可以通过传感器反馈从而达到记忆模式的效果。
%通过神经网络模型结合物理环境信息来确定控制方案的方法也已经被提出来~\cite{InformationDriven}~\cite{NovelPlasticityRule}~\cite{NeuroFuzzyBayesian},但是目前还无法通过具体实际的实验来验证。

\subsection{国内研究现状}
唐超权等提出了一种基于多模态CPG(中枢模式发生器)模型的控制策略~\cite{CPGenabling},该控制策略是基于预定义的预期速度值,利用当前速度和预期速度的关系来自调整输入信号进而诱导CPG模型做出运动的改变的多模态CPG模型。

吴晓东等人针对蛇形机器人呢自主无碰撞行为提出了用于转弯运动的幅度调制方法(AMM)的计算模型~\cite{Wu2010Autonomous}。利用能够使得蛇形机器人头部始终指向运动方向的头部导航运动模式提出了一种计算模型能够方便计算出关节角度与转角之间的关系,从而如果确定了无碰撞行为所需要的期望转角,就可以就散出模型的关节角度的变化。紧接着提出了基于传感器的神经系统网络实现的蛇形机器人无碰撞行为~\cite{Wu2011Sensor}。使用三个IR范围的传感器来获取障碍物信息,根据蛇形机器人的运动策略和行为基于神经元模型构建了信号反馈网络,然后将感知信号用作CPG振荡器输入的调整值。最后通过改变CPG网络中伸肌神经元或屈肌神经元的驱动输入,实现了蛇形机器人可以执行所需的转动动作以避开障碍物。

为了使蛇形机器人去自主适应未知的环境,即能够在未知的环境中做出正确的运动步态和姿态,使用嵌入式传感器来感知环境并根据传感器数据做出对应的环境感知之后做出相对应的控制算法的方法已被广泛使用~\cite{BalancingAndControl}~\cite{FeedbackControlOfSoft}~\cite{CPGenabling}~\cite{GaitBasedCompliant}。Rollinson和唐超权提出的控制方案的提出在相应场景下对机器人的控制比人类去单独控制每一个控制参数来控制机器人以及固有的运动方程控制方案确实更有效,且能运用更丰富灵活的步态运动。然而参数化步态和中央模式生成器~\cite{ijspeert2008central}这样的简单控制器虽然已经取得了一些成功,但使用这些控制器创建自主或自适应行为已经证明是困难的。

\section{本文的论文结构与章节安排}

\label{sec:arrangement}
本文包括如下五个章节,具体每个章节的内容安排如下所示:

第一章:引言,本章介绍了蛇形机器人在自适应运动控制方面的研究背景,意义,国内外研究现状及相关工作安排。

第二章:蛇形机器人结构,步态和控制方法,介绍了本文中采用的蛇形机器人设计结构以及本文实验中所使用的关节模块基于正交结构设计的蛇形机器人的控制方程。

第三章:研究方法介绍,详细地介绍了本文所采用的研究算法,介绍了算法的框架以及每个步骤的详细分析。

第四章:实验和结果展示,本章介绍了实验环境,展示了本文所采用的算法在蛇形机器人爬杆运动中的表现情况,包括数据分析,对照实验,算法性能分析。

第五章:总结与展望。总结本文的研究结论,同时根据目前理论和实验的不足之处,对下一步的研究工作做展望。

