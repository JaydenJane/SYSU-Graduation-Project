%%
% 摘要信息
% 本文档中前缀"c-"代表中文版字段, 前缀"e-"代表英文版字段
% 摘要内容应概括地反映出本论文的主要内容,主要说明本论文的研究目的、内容、方法、成果和结论。要突出本论文的创造性成果或新见解,不要与引言相 混淆。语言力求精练、准确,以 300—500 字为宜。
% 在摘要的下方另起一行,注明本文的关键词(3—5 个)。关键词是供检索用的主题词条,应采用能覆盖论文主要内容的通用技术词条(参照相应的技术术语 标准)。按词条的外延层次排列(外延大的排在前面)。摘要与关键词应在同一页。
% modifier: 黄俊杰(huangjj27, 349373001dc@gmail.com)
% update date: 2017-04-15
%%

\cabstract{
蛇形机器人是由多个可活动关节进行连接,具有超高冗余度的一种仿生机器人。然而,这种机器人的自主性运动是一个十分复杂的问题。本文提出了蛇形机器人爬杆运动中的一套自适应控制框架。通过对无监督训练的数据进行聚类和多参数的快速回归,本文提出的控制框架能够保证蛇形机器人在爬杆运动中对环境变化做出快速反应。实验结果表明了通过利用该控制框架,蛇形机器人在运动过程中选择新的控制参数进行调整的时间开销约为100ms,显然对于蛇形机器人在运动过程中能够以十分短的时间内做出正确的步态调整,并且蛇形机器人能够在本文提出的控制框架的控制下可以去爬升不同直径的杆。
}
% 中文关键词(每个关键词之间用“;”分开,最后一个关键词不打标点符号。)
\ckeywords{蛇形机器人;自动化爬杆运动;加权回归分析算法;机器学习}

\eabstract{
Snake-liked robots are a class of biomorphic hyper-redundant robots consist of many chain-connected active joint modules. Autonomy of this kind of robots is however a complex problem. This paper proposes an adaptive control framework for the pole climbing of snake robots. By clustering the unsupervised training data and fast multi-parameter regression, the framework proposed in this paper can help snake-liked robots rapidly react to environment changes. Experimental results show that, by using the framework proposed in this papaer, the overhead of choosing a control parameter and setting a new value for it is about 100 ms, which means that snake-liked robots can adjust the gaits correctly and fastly during the motions. What's more, snake-liked robots can climb poles with different diameters with the framework proposed in this paper.
}
% 英文文关键词(每个关键词之间用半角加空格分开, 最后一个关键词不打标点符号。)
\ekeywords{Snake-liked robots; autonomous pole climbing; weighted regression analysis; machine learning}

