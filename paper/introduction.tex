\section{Introduction}
% no \IEEEPARstart
%Snake-like robots generally consist of rigid links connected by joints which are usually fitted with actuators to control the locomoton to form a serial structure that mimic the skeleton of a real snake. With multiple joints and degree of freedom, snake-like robots have strong mobility in complex and unknown terrains\cite{Chirikjian1995The}. Therefore, they can complete tasks in many occasions, such as disaster rescuing\cite{DogAndSnake}, factory pipe maintenance\cite{ACMTutorial} and scientific exploration\cite{Kuwada2007Snake}. In the past decades, with the advent of orthogonal and universal joint structure\cite{1014757}\cite{Date2005Control}\cite{GaitBasedCompliant},  snake-like robots acquired the ability of three-dimensional space movement, which strengthens their mobility in varied topography. 

%Although the snake-like robots have high mobility in many environments, the control of snake-like robots is complicated. A variety of motions can be constructed to get the same motion effect in the same environment because of the features of snake-like robots like multiple joints and degree of freedom. Moreover, the movements of snake-like robots are difficult to be predicted in the unknown environment.

%Nowadays, there are two widely used control method. Firstly, we can pre-define the parameters to make robots move correctly in the known environment based on a widely used motion model\cite{HiroseSine} of snake-like robots. However, this method is not feasible for the unknown environment. Secondly, we can control the robot by tunning parameters artificially. While tunning parameters artifically is inefficient in unknown environments because it is difficult to tune the parameter accurately during robots' motion. In addition, controlling the robots is a professional work and tunning the parameters inaccurately will break the motion of snake-like robots. Therefore, autonomous mobility of robots is an expectation in robots' motion.
%Although the snake-like robots have high mobility in many environments, the movement of most snake-like robots in unknown environments is directly controlled by human, reducing the efficiency and limiting the implementation of snake-like robots. Therefore, it is an important and challenging task to give them the ability to move in complex environment without real-time human intervention, that is, the autonomous mobility.

%Autonomous mobility is an important and challenging task in robots' controlling.  Autonomous mobility requires high real-time performance to make robots respond to changes in the environment in time. What's more, High efficiency is critical in autonomous mobility. With high real-time performance and high efficiency, robots can make a rapid calculation when a mutation happens in the motion environment .   

%In this paper, we propose a novel experience-based control strategy, which autonomously determines the optimal control parameters of a snake-like robot in current environment by entropy variance\cite{WaveformEntropyVariance}\cite{EntropyandVarianceasRiskMeasure}\cite{UsingEntropyAndVariance}. By tunning the optimal control parameter with regression value through the weighted least squares algorithm\cite{gradientMethod}\cite{MSEestimates}, we achieve to make the robot gradually adapt itselt to the environment. The whole process of our control strategy can be divided into two parts:
%\begin{enumerate}
%	\item data acquisition and preprocessing,
%	\item real-time data feedback and multi-parameter regression control.
%\end{enumerate}
%\begin{figure}[!t]
%	\centering
%	\subfigure[Reality robot]{
%		\includegraphics[width=1.5in,height=0.75in]{fig/relative/realSnake}
%		\figlabel{fig:snake_like_robot_a}
%	}
%	\subfigure[Simulation robot]{
%		\includegraphics[width=1.5in,height=0.75in]{fig/relative/simulateSnake}
%		\figlabel{fig:snake_like_robot_b}
%	}
%	\caption{The Snake-like robot}
%	\figlabel{fig:snake_like_robot}
%\end{figure}
%In the first part, we collect data that reflecting the dynamic states of a snake-like robot in varies environments and then form a training set. The clustering algorithm\cite{Cluseter_ICT}\cite{KmeansAndDeepLearning} is adopted in this part to reduce the time overhead of feed-back control. In the second part, we apply regression and feed-back control based on previous training data of the motion of robots. The multi-parameter regression is also transformed into an unit regression problem for higher efficiency. We adopt the entropy variance\cite{WaveformEntropyVariance}\cite{EntropyandVarianceasRiskMeasure}\cite{UsingEntropyAndVariance} as the criterion for accessing the effect of the control parameters on the motion. Then the most sensitive parameter which has the greatest effect on motion is selected to perform the unit regression. By transforming the multiple regression into the  unit regression and combining the weighted least squares algorithm\cite{gradientMethod}\cite{MSEestimates},  we finally get a fitted regression function and obtain the value of the optimal parameter.
%To evaluate the effectiveness of the proposed framework, we also apply it to the our existing snake-like robot platform, which is shown in~\figref{fig:snake_like_robot_a} and~\figref{fig:snake_like_robot_b}. Under our framework, the robot climbs pipes having different diameters and the performance of climbing is reported. In summary, the contributions of this paper are: 
%\begin{itemize}
%	\item A new framework for robot adaptive control is proposed.
% 	      This framework greatly reduces the amount of  effort of the regression calculation 
%           by adopting clustering algorithm for data preprocessing and the classification of 
%           real-time data.
%	\item A transformation of multiple regression problem is proposed to obtain the optimal
%          control parameters in a real-time manner.
%         The entropy variance is utilized to select the parameter which is the most sensitive 
%         in current state. Then, by regressing this parameter instead of all the parameters, the 
%         multiple regression is transformed into an unit regression problem.
%    \item A set of case studies is conducted and the results demonstrate
%    that our proposed strategy is effectiveness in accomplishing autonomous movement for snake-like robots.
%\end{itemize}
%The robot has an orthogonal structure with 18 joints. Each joint has a rotation axis which
%is perpendicular to the body. The rotation axes of two adjacent joints are orthogonal with each other,
%thus forming a three-dimensional space movement. a set of sensors such as accelerometer and gyro
%is embedded in each joint to realize the feedback control.

%The rest of this paper is organised as follows. Early preparations are demonstrated in the Section \uppercase\expandafter{\romannumeral2}. In Section \uppercase\expandafter{\romannumeral3}, proposed strategy is explained in detailed. And then Simulation works are illustrated in Section \uppercase\expandafter{\romannumeral4}. Section \uppercase\expandafter{\romannumeral5} shows the conclusion and our future prospects about the proposed strategy.

%The simulation environment is the pipe climbing, which can be used in cable detection of cable-stayed bridge. In the first part, we collect data in pipes with different diameter as training set. To reduce time of calculation in feed-back control, we adopt clustering algorithm\cite{Cluseter_ICT}\cite{KmeansAndDeepLearning}. In the second part, we apply regression and feed-back control based on previous training data on the motion of robots. We utilize the entropy variance\cite{WaveformEntropyVariance}\cite{EntropyandVarianceasRiskMeasure}\cite{UsingEntropyAndVariance} to measure the effect of the control parameters on the motion and select the most sensitive parameter which has the greatest effect on motion. By transforming the multiple regression into the unit regression and combining the weighted least squares algorithm\cite{gradientMethod}\cite{MSEestimates}, we finally get a fitted regression function and modify the most sensitive parameter.

Snake robots are a class of biomorphic hyper-redundant
robots~\cite{Chirikjian1995The}, designed to imitate the snake
creatures’ biological limbless locomotion with outstanding rapidity,
stability, and diversity in wild environment. Typically, these
snake robots consist of many chain-connected active joint
modules, giving them kinematic versatility, like bending, stretching,
and crispation.  A considerable number of these robots have been
implemented in various fields, such as disaster rescuing, factory
maintenance, and terrorism surveillance.

In order to better complete predefined tasks, snake robots are
required to obtain capabilities of moving autonomously and behaving
self-adaptively~\cite{Liljeb2013Snake}, e.g., making decision when,
where, and how to move based on different situations of itself and
environment. Automatically adapting to the environment is, however,
not easy. The reasons are multi-folds. First, due to the redundant
degrees of freedom, the locomotion of the snake robot is
difficult to model, especially including the complex interaction
between the robot and the environment.  The corresponding control for
this multi-degree locomotion is even more complicate. Second, when
encountering an environment which is not known beforehand, how to
decide a suitable control strategy is obvious. Even for a given
control strategy, how to decide its parameters is not
straightforward. Third, the runtime procedure of deciding the control
strategy and its corresponding parameters must be efficient, i.e., in
real time. Otherwise, the desired locomotion of the robot will most
probably fail.

To let the robots adapt to an unknown environment, methods that use
the sensors to perceive the environment and to embed the environment
perception rules has been widely
used~\cite{CPGenabling,GaitBasedCompliant,BalancingAndControl,FeedbackControlOfSoft}. Tang
et al. proposed a control strategy based on CPG(central pattern
generator) model~\cite{CPGenabling}. Rollinson et al. proposed a
snake robot adaptive control based on state
estimation~\cite{GaitBasedCompliant}.  As the model in these methods is
a gradient model based on state estimation, these they are not suited
to the mutation environment.
There are researchers that have proposed neural network models
combined with physical environment information to determine the
control
scheme~\cite{InformationDriven,NovelPlasticityRule,MissileSystems,NeuroFuzzyBayesian}. These
models are only the control suggestion and may not make a good effect
on real-time motion of the robots.

In this paper, a new framework for adaptive control of snake robots is
proposed. Based on data that are collected off-line by unsupervised
training, our framework can efficiently adapt new control parameters
for environment changes by fast regression. Specifically, we target
pole climbing, which demands a fast response to the environment
changes. In this scenario, if the control cannot adapt on-time, the
robot will fall.  To evaluate the effectiveness of the proposed
approach, we conduct experiments on a snake robot on different
poles. The experimental results show that the overhead of runtime
adaptation is slightly more than 100\,ms and snake robot can fast
adapt to different poles and climb along the poles smoothly. The
contributions of this paper are:
\begin{itemize}
\item A adaptive control framework for the pole climbing of snake
  robots. For the offline unsupervised training, a k-means++ method is
  used to cluster the collected data. During runtime, the Z-axis
  velocity of the robot is used as a feed-back signal to continuously
  adapt new control parameters of the robot.
\item A transformation of multiple regression problem is proposed to
  obtain the optimal control parameters in a real-time manner.  The
  entropy variance is utilized to select the parameter which is the
  most sensitive in current state. Then, by regressing this parameter
  instead of all the parameters, the multiple regression is
  transformed into an unit regression problem.
\item A set of case studies is conducted and the results demonstrate
  that our proposed strategy is effectiveness in accomplishing
  autonomous movement for snake robots.
\end{itemize}

The rest of this paper is organised as follows. The relative work is shown in Section \uppercase\expandafter{\romannumeral2} and an overview of our work is
demonstrated in the Section \uppercase\expandafter{\romannumeral3}. In
Section \uppercase\expandafter{\romannumeral4} and \uppercase\expandafter{\romannumeral5}, proposed strategy is
explained in detailed. And then Simulation works are illustrated in
Section \uppercase\expandafter{\romannumeral6}. Section
\uppercase\expandafter{\romannumeral7} shows the conclusion and our
future prospects about the proposed strategy.